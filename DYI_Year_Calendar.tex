%%%%%%%%%%%%%%%
% YEAR CALENDAR
%
% LAYOUT: 3 months per row, 4 rows (one page -- odd/right)
%					   <year>
%		---------------------------------
%		MONTH <space> MONTH <space> MONTH
%		MONTH <space> MONTH <space> MONTH
%		MONTH <space> MONTH <space> MONTH
%		MONTH <space> MONTH <space> MONTH
%		---------------------------------
%		---------------------------------
%		---------------------------------

% Settings-----------------------------------------------------------

% percentage (0.0 - 1.0) taken by the 3 months tables (3 months in a row) on the Year Calendar
% (the rest is equally split between 2 inter month-column spaces)
\newcommand\percentMonthColTWidthYC{0.95}

% calculate width of the month tabular* on the Year Calendar
\newlength{\MonthTblWidthYC}
\setlength{\MonthTblWidthYC}{( \textwidth * \real{\percentMonthColTWidthYC} ) / 3}

% calculate weekday column width inside the month tabular* on the Year Calendar
\newlength{\WkdayColWidthMonthTblYC}
\setlength{\WkdayColWidthMonthTblYC}{\MonthTblWidthYC / 7}

% Column Types for Months Tables on the Year Calendar
% Notes: - using 'tabular*' so use '@{\extracolsep\fill}')
%        - align to the 0.5em of right (supress the intercolumn space for better control)
%	 - color affects space (inserts struts in current font), put it after font change
% There are 4 types of columns
% - Column 1 (marked as weekend or weekday depending on week_starts_on_Monday value)
% This is output to DYI_i18n.tex
% - Column 2-5 (always weekday)
\newcolumntype{B}{>{\hfill\normalfont\footnotesize}p{\WkdayColWidthMonthTblYC}<{\hspace*{0.5em}}@{\extracolsep\fill}}
% - Column 6 (marked as weekend or weekday depending on week_starts_on_Monday value)
% This is output to DYI_i18n.tex
% - Column 7 (always weekend)
\newcolumntype{D}{>{\hfill\normalfont\footnotesize\color{WeekendDay}}p{\WkdayColWidthMonthTblYC}<{\hspace*{0.5em}}}

\newcommand{\MonthTblYC}[1]{%
	\begin{tabular*}{\MonthTblWidthYC}[t]{@{}A*{4}{B}CD@{}}
		\rowcolor{HeadSubBg}
		\WkdayTblRow{\bfseries\tiny} \\
		#1{\hfill}{\color{WeekendDay}\hfill}
	\end{tabular*}}

\newcommand{\MonthNameRowYC}[3]{%
	\rowcolor{HeadMainBg}
	\vstrut{1.1em}
	\color{white}#1 & \color{white}#2 & \color{white}#3}

% Start--------------------------------------------------------------

\hfill{\bfseries\Huge\MyYear}\par
\nointerlineskip\vspace{0.5em}
\rule{\textwidth}{1pt}

\nointerlineskip\vspace{0.5em}

\begin{tabular*}{\textwidth}{@{}>{\bfseries}c@{\extracolsep\fill}>{\bfseries}c@{\extracolsep\fill}>{\bfseries}c@{}}
	\MonthNameRowYC{\January}{\February}{\March}    \\
	\MonthTblYC{\MonthTblJan}  & \MonthTblYC{\MonthTblFeb}  & \MonthTblYC{\MonthTblMar} \\
	\MonthNameRowYC{\April}{\May}{\June}            \\
	\MonthTblYC{\MonthTblApr}  & \MonthTblYC{\MonthTblMay}  & \MonthTblYC{\MonthTblJun} \\
	\MonthNameRowYC{\July}{\August}{\September}     \\
	\MonthTblYC{\MonthTblJul}  & \MonthTblYC{\MonthTblAug}  & \MonthTblYC{\MonthTblSep} \\
	\MonthNameRowYC{\October}{\November}{\December} \\
	\MonthTblYC{\MonthTblOct}  & \MonthTblYC{\MonthTblNov}  & \MonthTblYC{\MonthTblDec}
\end{tabular*}

% fill the rest with lines, for notes
\vspace{2em}
{\color{WriteBgMain}
\rule{\textwidth}{1pt}\par
\rule{\textwidth}{1pt}\par
\rule{\textwidth}{1pt}\par
\rule{\textwidth}{1pt}\par
\rule{\textwidth}{1pt}\par
\rule{\textwidth}{1pt}\par
\rule{\textwidth}{1pt}\par
\rule{\textwidth}{1pt}\par}

